% Options for packages loaded elsewhere
\PassOptionsToPackage{unicode}{hyperref}
\PassOptionsToPackage{hyphens}{url}
%
\documentclass[
]{article}
\usepackage{lmodern}
\usepackage{amssymb,amsmath}
\usepackage{ifxetex,ifluatex}
\ifnum 0\ifxetex 1\fi\ifluatex 1\fi=0 % if pdftex
  \usepackage[T1]{fontenc}
  \usepackage[utf8]{inputenc}
  \usepackage{textcomp} % provide euro and other symbols
\else % if luatex or xetex
  \usepackage{unicode-math}
  \defaultfontfeatures{Scale=MatchLowercase}
  \defaultfontfeatures[\rmfamily]{Ligatures=TeX,Scale=1}
\fi
% Use upquote if available, for straight quotes in verbatim environments
\IfFileExists{upquote.sty}{\usepackage{upquote}}{}
\IfFileExists{microtype.sty}{% use microtype if available
  \usepackage[]{microtype}
  \UseMicrotypeSet[protrusion]{basicmath} % disable protrusion for tt fonts
}{}
\makeatletter
\@ifundefined{KOMAClassName}{% if non-KOMA class
  \IfFileExists{parskip.sty}{%
    \usepackage{parskip}
  }{% else
    \setlength{\parindent}{0pt}
    \setlength{\parskip}{6pt plus 2pt minus 1pt}}
}{% if KOMA class
  \KOMAoptions{parskip=half}}
\makeatother
\usepackage{xcolor}
\IfFileExists{xurl.sty}{\usepackage{xurl}}{} % add URL line breaks if available
\IfFileExists{bookmark.sty}{\usepackage{bookmark}}{\usepackage{hyperref}}
\hypersetup{
  pdftitle={Pre-workshop Installation Instructions},
  hidelinks,
  pdfcreator={LaTeX via pandoc}}
\urlstyle{same} % disable monospaced font for URLs
\usepackage[margin=1in]{geometry}
\usepackage{color}
\usepackage{fancyvrb}
\newcommand{\VerbBar}{|}
\newcommand{\VERB}{\Verb[commandchars=\\\{\}]}
\DefineVerbatimEnvironment{Highlighting}{Verbatim}{commandchars=\\\{\}}
% Add ',fontsize=\small' for more characters per line
\usepackage{framed}
\definecolor{shadecolor}{RGB}{248,248,248}
\newenvironment{Shaded}{\begin{snugshade}}{\end{snugshade}}
\newcommand{\AlertTok}[1]{\textcolor[rgb]{0.94,0.16,0.16}{#1}}
\newcommand{\AnnotationTok}[1]{\textcolor[rgb]{0.56,0.35,0.01}{\textbf{\textit{#1}}}}
\newcommand{\AttributeTok}[1]{\textcolor[rgb]{0.77,0.63,0.00}{#1}}
\newcommand{\BaseNTok}[1]{\textcolor[rgb]{0.00,0.00,0.81}{#1}}
\newcommand{\BuiltInTok}[1]{#1}
\newcommand{\CharTok}[1]{\textcolor[rgb]{0.31,0.60,0.02}{#1}}
\newcommand{\CommentTok}[1]{\textcolor[rgb]{0.56,0.35,0.01}{\textit{#1}}}
\newcommand{\CommentVarTok}[1]{\textcolor[rgb]{0.56,0.35,0.01}{\textbf{\textit{#1}}}}
\newcommand{\ConstantTok}[1]{\textcolor[rgb]{0.00,0.00,0.00}{#1}}
\newcommand{\ControlFlowTok}[1]{\textcolor[rgb]{0.13,0.29,0.53}{\textbf{#1}}}
\newcommand{\DataTypeTok}[1]{\textcolor[rgb]{0.13,0.29,0.53}{#1}}
\newcommand{\DecValTok}[1]{\textcolor[rgb]{0.00,0.00,0.81}{#1}}
\newcommand{\DocumentationTok}[1]{\textcolor[rgb]{0.56,0.35,0.01}{\textbf{\textit{#1}}}}
\newcommand{\ErrorTok}[1]{\textcolor[rgb]{0.64,0.00,0.00}{\textbf{#1}}}
\newcommand{\ExtensionTok}[1]{#1}
\newcommand{\FloatTok}[1]{\textcolor[rgb]{0.00,0.00,0.81}{#1}}
\newcommand{\FunctionTok}[1]{\textcolor[rgb]{0.00,0.00,0.00}{#1}}
\newcommand{\ImportTok}[1]{#1}
\newcommand{\InformationTok}[1]{\textcolor[rgb]{0.56,0.35,0.01}{\textbf{\textit{#1}}}}
\newcommand{\KeywordTok}[1]{\textcolor[rgb]{0.13,0.29,0.53}{\textbf{#1}}}
\newcommand{\NormalTok}[1]{#1}
\newcommand{\OperatorTok}[1]{\textcolor[rgb]{0.81,0.36,0.00}{\textbf{#1}}}
\newcommand{\OtherTok}[1]{\textcolor[rgb]{0.56,0.35,0.01}{#1}}
\newcommand{\PreprocessorTok}[1]{\textcolor[rgb]{0.56,0.35,0.01}{\textit{#1}}}
\newcommand{\RegionMarkerTok}[1]{#1}
\newcommand{\SpecialCharTok}[1]{\textcolor[rgb]{0.00,0.00,0.00}{#1}}
\newcommand{\SpecialStringTok}[1]{\textcolor[rgb]{0.31,0.60,0.02}{#1}}
\newcommand{\StringTok}[1]{\textcolor[rgb]{0.31,0.60,0.02}{#1}}
\newcommand{\VariableTok}[1]{\textcolor[rgb]{0.00,0.00,0.00}{#1}}
\newcommand{\VerbatimStringTok}[1]{\textcolor[rgb]{0.31,0.60,0.02}{#1}}
\newcommand{\WarningTok}[1]{\textcolor[rgb]{0.56,0.35,0.01}{\textbf{\textit{#1}}}}
\usepackage{graphicx,grffile}
\makeatletter
\def\maxwidth{\ifdim\Gin@nat@width>\linewidth\linewidth\else\Gin@nat@width\fi}
\def\maxheight{\ifdim\Gin@nat@height>\textheight\textheight\else\Gin@nat@height\fi}
\makeatother
% Scale images if necessary, so that they will not overflow the page
% margins by default, and it is still possible to overwrite the defaults
% using explicit options in \includegraphics[width, height, ...]{}
\setkeys{Gin}{width=\maxwidth,height=\maxheight,keepaspectratio}
% Set default figure placement to htbp
\makeatletter
\def\fps@figure{htbp}
\makeatother
\setlength{\emergencystretch}{3em} % prevent overfull lines
\providecommand{\tightlist}{%
  \setlength{\itemsep}{0pt}\setlength{\parskip}{0pt}}
\setcounter{secnumdepth}{-\maxdimen} % remove section numbering

\title{Pre-workshop Installation Instructions}
\author{}
\date{\vspace{-2.5em}}

\begin{document}
\maketitle

Contents borrowed and modified from
\href{https://uvastatlab.github.io/phdplus/intror.html}{UVA's Data
Science Essentials in R series}

\hypertarget{before-the-first-session}{%
\subsection{Before the first session}\label{before-the-first-session}}

To participate in the R workshop, please bring a laptop with R and
RStudio installed. We recommend that you have the latest version of R
(3.6.*), the \texttt{tidyverse} package (1.2), the \texttt{learnr}
package (0.9), and the \texttt{here} package (0.1). You will then need
to use the \texttt{devtools} package to download the \texttt{workr}
package we developed for our workshop tutorials. You need to have
RStudio installed, but it is less crucial that you are using the most
recent version (1.2).

\textbf{Do you already have R and RStudio installed?}

\begin{itemize}
\tightlist
\item
  No - follow the instructions for ``I do not have R installed''\\
\item
  Yes - follow the instructions for ``I have R installed''
\end{itemize}

\hypertarget{i-do-not-have-r-installed}{%
\subsection{``I do not have R
installed''}\label{i-do-not-have-r-installed}}

You should install R, RStudio, \texttt{tidyverse}, \texttt{learnr}, and
\texttt{here}.

\hypertarget{installing-r}{%
\subsubsection{Installing R}\label{installing-r}}

\hypertarget{windows}{%
\paragraph{Windows:}\label{windows}}

\begin{enumerate}
\def\labelenumi{\arabic{enumi}.}
\tightlist
\item
  Go to \url{https://cloud.r-project.org/bin/windows/base/}
\item
  Click the ``Download R 3.6.1 for Windows'' link. (Or whatever the
  newest version is)
\item
  When the file finishes downloading, double-click to install. You
  should be able to click ``Next'' to all dialogs to finish the
  installation.
\end{enumerate}

\hypertarget{mac}{%
\paragraph{Mac:}\label{mac}}

\begin{enumerate}
\def\labelenumi{\arabic{enumi}.}
\tightlist
\item
  Go to \url{https://cloud.r-project.org/bin/macosx/}
\item
  Click the link ``R-3.6.1.pkg''
\item
  When the file finishes downloading, double-click to install. You
  should be able to click ``Next'' to all dialogs to finish the
  installation.
\end{enumerate}

\hypertarget{linux}{%
\paragraph{Linux:}\label{linux}}

For any adventurous Linux users in our group follow this guide
(\url{https://github.com/duckmayr/install-update-r-on-linux}) to
install/upgrade to the most recent version of R on Ubuntu (18.04) or
Mint (19).

\hypertarget{installing-rstudio}{%
\subsubsection{Installing RStudio}\label{installing-rstudio}}

\begin{enumerate}
\def\labelenumi{\arabic{enumi}.}
\tightlist
\item
  Go to
  \href{https://www.rstudio.com/products/rstudio/download/\#download}{the
  RStudio download page}.
\item
  Under ``Installers for Supported Platforms'' select the appropriate
  installer for your operating system
\item
  When the file finishes downloading, double-click to install. You
  should be able to click ``Next'' to all dialogs to finish the
  installation.
\end{enumerate}

\hypertarget{installing-packages}{%
\subsubsection{Installing packages}\label{installing-packages}}

Skip ahead to the \textbf{Installing and updating packages} section for
instructions on how to install the necessary packages for our workshop.

\hypertarget{i-have-r-installed}{%
\subsection{``I have R installed''}\label{i-have-r-installed}}

The workshops run more smoothly when everyone is using the same version
of R, \texttt{tidyverse}, and \texttt{learnr}. Please update R,
\texttt{tidyverse}, and \texttt{learnr} if necessary (and less
crucially, RStudio).

\hypertarget{verify-r-version}{%
\subsubsection{Verify R version}\label{verify-r-version}}

Open RStudio. At the top of the Console you will see session info. The
first line tells you which version of R you are using. If RStudio is
already open and you're deep in a session, type
\texttt{R.version.string} in the console and enter to print out the R
version.

Do you have R version 3.6.* installed?

\begin{itemize}
\tightlist
\item
  No - follow the instructions for ``Updating R''
\item
  Yes - Great! Do you have \texttt{tidyverse}, \texttt{learnr}, and
  \texttt{here} installed?

  \begin{itemize}
  \tightlist
  \item
    No or I don't know - See ``Installing \texttt{tidyverse}''
  \item
    Yes - Great! Go to Go to Tools \textgreater{} Check for Package
    Updates. If there's an update available for \texttt{tidyverse},
    install it.
  \end{itemize}
\end{itemize}

\hypertarget{updating-rrstudio}{%
\subsubsection{Updating R/RStudio}\label{updating-rrstudio}}

\hypertarget{windows-1}{%
\paragraph{Windows}\label{windows-1}}

To update R on Windows, try using the package \texttt{installr} (only
for Windows).

\begin{enumerate}
\def\labelenumi{\arabic{enumi}.}
\tightlist
\item
  Install and load installr:
\end{enumerate}

\begin{Shaded}
\begin{Highlighting}[]
\KeywordTok{install.packages}\NormalTok{(}\StringTok{"installr"}\NormalTok{)}
\KeywordTok{library}\NormalTok{(installr)}
\end{Highlighting}
\end{Shaded}

\begin{enumerate}
\def\labelenumi{\arabic{enumi}.}
\setcounter{enumi}{1}
\tightlist
\item
  Call \texttt{updateR()} function. This will start the updating process
  of your R installation by: ``finding the latest R version, downloading
  it, running the installer, deleting the installation file, copy and
  updating old packages to the new R installation.''\\
\item
  From within RStudio, go to Help \textgreater{} Check for Updates to
  install newer version of RStudio (if available, optional).
\end{enumerate}

\hypertarget{mac-1}{%
\paragraph{Mac}\label{mac-1}}

On Mac, you can simply download and install the newest version of R.
When you restart RStudio, it will use the updated version of R.

\begin{enumerate}
\def\labelenumi{\arabic{enumi}.}
\tightlist
\item
  Go to \url{https://cloud.r-project.org/bin/macosx/}\\
\item
  Click the link ``R-3.6.1.pkg'' (or whatever the latest version is)
\item
  When the file finishes downloading, double-click to install. You
  should be able to click ``Next'' to all dialogs to finish the
  installation.\\
\item
  From within RStudio, go to Help \textgreater{} Check for Updates to
  install newer version of RStudio (if available, optional).
\end{enumerate}

\hypertarget{linux-1}{%
\paragraph{Linux:}\label{linux-1}}

Again, for any adventurous Linux users in our group follow this guide
(\url{https://github.com/duckmayr/install-update-r-on-linux}) to
install/upgrade to the most recent version of R on Ubuntu (18.04) or
Mint (19).

\hypertarget{installing-and-updating-packages}{%
\subsection{Installing and updating
packages}\label{installing-and-updating-packages}}

\hypertarget{installing-tidyverse-learnr-and-here}{%
\subsubsection{\texorpdfstring{Installing \texttt{tidyverse},
\texttt{learnr}, and
\texttt{here}}{Installing tidyverse, learnr, and here}}\label{installing-tidyverse-learnr-and-here}}

\begin{enumerate}
\def\labelenumi{\arabic{enumi}.}
\tightlist
\item
  Open RStudio
\item
  Go to \texttt{Tools} \textgreater{} \texttt{Install\ Packages}
\item
  Enter \texttt{tidyverse}
\item
  Select \texttt{Install}
\end{enumerate}

Follow the same protocol for installing the rest of the packages, but
replace \texttt{tidyverse} with the package names. Alternatively, you
can install packages by running the command \texttt{install.packages()}
in your console. You can install multiple packages as once by combining
package names into a vector as follows:

\begin{Shaded}
\begin{Highlighting}[]
\KeywordTok{install.packages}\NormalTok{(}\KeywordTok{c}\NormalTok{(}\StringTok{"tidyverse"}\NormalTok{,}\StringTok{"learnr"}\NormalTok{, }\StringTok{"here"}\NormalTok{))}
\end{Highlighting}
\end{Shaded}

\hypertarget{installing-workr-using-the-devtools-package}{%
\subsubsection{\texorpdfstring{Installing \texttt{workr} using the
\texttt{devtools}
package}{Installing workr using the devtools package}}\label{installing-workr-using-the-devtools-package}}

While there are many useful packages hosted on the CRAN (the global
repository of R packages queried using \texttt{install.packages()}), you
may be interested in using packages that are still in development. Many
of these packages are shared on GitHub and can be downloaded using
\texttt{install\_github()}, a function of the \texttt{devtools} package.
Below is an example of how to install the \texttt{workr} package using
\texttt{devtools}.

\begin{Shaded}
\begin{Highlighting}[]
\KeywordTok{install.packages}\NormalTok{(}\StringTok{"devtools"}\NormalTok{)}
\NormalTok{devtools}\OperatorTok{::}\KeywordTok{install_github}\NormalTok{(}\StringTok{"https://github.com/natalieoshea/workr.git"}\NormalTok{) }
\CommentTok{## If it asks to install or update packages, be sure to select "CRAN packages only"}
\end{Highlighting}
\end{Shaded}

Including \texttt{devtools:::} before \texttt{install\_github()} allows
you to call the function without having to load the entire library.
However, if you are planning to use additional functions from the
\texttt{devtools} package you may want to fully load the package
(\texttt{library(devtools)}--\textgreater{} \texttt{install\_github()})

\hypertarget{check-for-package-updates}{%
\subsubsection{Check for package
updates}\label{check-for-package-updates}}

If you already have all packages installed you can check for updates to
CRAN packages using the \texttt{update.packages()} command or by going
to \texttt{Tools}--\textgreater{} \texttt{Check\ for\ Package\ Updates}.
Packages installed from GitHub can be updated using \texttt{devtools}
with the \texttt{update\_packages()} command. For example, to check for
updates to one of our tutorials, be sure to run
\texttt{devtools::update\_packages(workr)} before you begin.

\end{document}
